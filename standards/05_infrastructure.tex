\chapter{Infrastructure}

\section{Physical infrastructure}



\subsection{Cash management (bankless)}

* All corporate cash must be physically segregated from none personal
cash

* All movement of corporate funds into an out of physical cash must be
recorded

* Physical cash is used only for corporate operations.  Twofish will
not handle physical cash for clients.

* The location of all cash must be physically inventoried

\subsection{Petty cash}
* The company may hold up to HKD 20k in non-secure facilities and in
form of petty cash.

\subsection{Segreated cash}
* Amounts up to HKD 200k may be held in a safe in a non-secure
location.  

\subsection{Physically secure facility}
* Amounts over HKD 200K must be held in a physically separate secure
location.

\subsection{Cash management}
* The company should hold two weeks operational reserve in the form of
paper fiat.

* The company should hold the remaining in the form of two

* The company should at all times have at least two contacts that can
convert paper cash to and from virtual assets

\section{Cybersecurity}

\subsection{Location of servers}
The location of the servers will be reported to the 

\subsection{Administrative access}
Administrative access is allowed only for devops work.  No trading is
allowed outside of the standard trading interfaces.

\begin{itemize}
\item The location of the exchange server is hidden behind a load
  balancer or cloud server.  The IP address of to allow access is not
  publically available
\item Direct root access to the server is not allowed, and shell access to
  the exchange server is only through passwordless tokens
\item (TODO) There should be a two step login process by which the
  user logs in first to a proxy server, and then from the proxy server
  undertakes a second login to the exchange server
\end{itemize}

\section{Software deployment process}
\subsection{Open Source only software}

The firm has a policy of only running open source software.  All
software which is run on productions backend services is publically
available via the github repository for the company.

\subsection{Logs}
The technology department should maintain a log of incidents and
events.  

\subsection{New version deployment}
The process for deploying a new version of OpenCEX:
\begin{itemize}
  \item New code deployments should generally be done on the weekend
  \item Deployments must be signed off by the chief technology officer
    or his delegate
  \item The deployment officer will examine the source code and verify
    that it is the correct version
  \item Code is downloaded from OpenCEX main site and merged with
    local changes on github
  \item Code is deployed on test server and run for two to three days.
  \item Before there is a server upgrade the directory containing the
    server engine will be backed up
  \item When there are no anomalies the code will upgraded on the
    production server
\end{itemize}
    
\subsection{Vendor patches}
The servers are operating under Linux Ubuntu.  Vendor upgrades should
be made as soon as possible but in no event more than one week after
the vendor upgrade is released.

\subsection{Server reboot}
The server should be reboot from a cold start at least once a week.

\subsection{Security issues}
The firm will work closely with vendors to identify and close security
issues.  Security related bugs will be addressed through standard
``responsible'' disclosure mechanism.

\subsection{Backups}


